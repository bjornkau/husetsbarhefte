\subsection{Vodka}
\subsubsection{Screwdriver}
\begin{itemize}
    \item[Rating (BK)] 1/10
    \item[Glass] Highball
    \item[Served] On the rocks
    \item Vodka 50ml
    \item Appelsinjus 100ml
\end{itemize}
Bygg i glasset.

\subsubsection{Cosmopolitan}
Drink gjort populær av TV-serie, trenger vi si mer? Er også veldig god, om man gjør det riktig.
\begin{itemize}
    \item[Rating (BK)] 9/10
    \item[Glass] Coup / Martiniglass
    \item[Served] Up
    \item Citron Vodka 40ml
    \item Cointreau 15ml
    \item Tranebærjuice 30ml
    \item Limejuice 15ml
    \item[Garnity] Limehjul
\end{itemize}
Shake og dobblestrain til glass.

\subsubsection{Lollipop}
Husets egen favorittdrink, men her med Bjørn's take på den. Note: Jeg synes denne ofte er for søt, så jeg lager den med mindre sukker. 
\begin{itemize}
    \item[Rating (BK)] 6/10
    \item[Glass] Highball
    \item[Served] On The Rocks
    \item Citron Vodka 10ml
    \item Jordbærlikør 30ml
    \item Bringebærsirup 20ml 
    \item Sitronsaft 20ml
    \item Solo
    \item[Garnity] Jordbær
\end{itemize}
Shake og strain i glass, topp med Solo.

\subsubsection{Vodka Martini}
\label{drink:v_martini}
"Shaken, not Stirred" - en av filmverdens mest kjente linjer. Mange forbinner denne cocktailen med James Bond, men det er ikke en original cocktial fra Ian Flemming. Det derimot, er Vesper (side \pageref{drink:vesper}), kjent fra Casino Royale.Eneste grunnen til å shake denne er om man vil vanne den ut mer. Anbefalingen er å røre den.
\begin{itemize}
    \item[Rating (BK)] 3/10
    \item[Glass] Coup / Martiniglass
    \item[Served] Up
    \item Gin 60ml
    \item Tør Vermouth 10ml
    \item[Garnityr] Sitronskall / Oliven
\end{itemize}
Rør, strain i glass.

\subsubsection{Espresso Martini}
Disclaimer, her må man nesten være veldig glad i kaffe før man lager den, den smaker veldig kaffe. Det kan også hjelpe å kjøle ned espressoen før man lager den, for da blir det mindre utvanning. IBA spesifiserer at man skal bruke Kahlúa som likør, men her kan man helt fint prøve flere.
\begin{itemize}
    \item[Rating (BK)] 8/10
    \item[Glass] Coup / Martiniglass
    \item[Served] Up
    \item Vodka 50ml
    \item Kaffelikør 10ml
    \item Espresso 40 ml
    \item Simple sirup (etter smak)
    \item[Garnityr] Kaffebønner
\end{itemize}
Shake, dobbelstrain til glass.

\subsubsection{Moscow Mule}
Denne har jeg sett flere variasjoner på, men i all hovedsak er det sprit, surt og mixer som gjelder, og ikke noe fancy sirup i bruk. Anbefaler også å ha en god ingefærøl, Fevertree og Bareksten er personlige favoritter.
\begin{itemize}
    \item[Rating (BK)] 8/10
    \item[Glass] Koppermugge
    \item[Served] One the Rocks / Crushed Ice
    \item Vodka 2oz (~60ml)
    \item Limejuice .75oz (~22.5ml)
    \item Ingefærøl
    \item[Garnityr] Limehjul
\end{itemize}
Bygg i glasset, gi en kort omrøring. Server med sugerør.
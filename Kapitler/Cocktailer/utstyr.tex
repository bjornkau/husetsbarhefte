\section{Utstyr}
\label{sec:utstyr}
\subsection{Målebeger}
Dette er kanskje det viktigste utstyret som finnes. Målebegere kommer i alle rare størrelser og fasonger. Personlig liker jeg det som kalles japansk stil jiggere \todo{Legg til bilde av jigger} best. Det som er greit å vite er hvor store målebegerene sine er, slik at man ikke måler feil når man mixer drinker. Om man skal ha målebegere med imperial aller metric måleenhter går litt ut på ett, så lenge man kan regne mellom dem så kan man lage det meste. 

\subsection{Shakere}
Shakere er samlebregepet på et ymse mengde med utstyr som er designet for å riste sammen ingredienser. Her og finnes det flere forskjellige typer, som har sine fordeler og ulemper. Det viktigste er at de er tette, og at de får plass til drinken man lager.
\subsubsection{Bostonshaker}
Dette er det vi som oftest bruker på Huset. Boston shakere kommer i to deler, et stort tin, og et mindre tin, eller et glass. Fordeler med disse er at de er større, og ofte er enklere å bruke. Ulemper er at de ikke alltid er like godt forseglet, og at er de av et for mykt metall er de veldig kjipe å ha med å gjøre. De har heller ikke innebygget strainer.\todo{legg til bilde av boston}
\subsubsection{Cobblershaker}
Cobblershakeren, eller 3-dels-shakeren er visstnok veldig populær blandt bartender i Japan. Denne kommer i tre deler, og har en innebygget strainer. De er ofte mindre enn Boston varianten, men det er ofte ikke et problem, om man lager en og en drink. Disse er ofte det første man skaffer seg av shakere, jeg gjorde det selv, og de er etter min erfaring, lettere å få tak i. De eneste ulempene er at de kan krympe når de blir kalde, og bli vanskelige å dele når man skal ha drinken ut av shakeren.\todo{legg til bilde av cobbler}

\subsection{Mixeglass}
Mixeglass brukes når man lager rørte drinker som ikke lages direkte i glasset. Et godt eksempel er Manhattan (oppskrift side \pageref{drink:manhattan}). En hvilken som helst beholder som har plass til drinken og is fungerer som mixeglass. Det finnes en hel del fancy glass på markedet, men et stort et holder lenge. Man kan også bruke deler av en shaker som mixeglass.

\subsection{Strainere/siler}
Siler brukes for å fjerne større partikler/rester av ingredienser/små isbiter som vi ikke vil ha i drinkene våre. Det er tre siler som brukes mye i drinkverden, det er hawthorne, julip og tesiler. Av disse er det hawthore og tesil som er greiest å ha, da hawthorne kan ta mesteparten av jobben til en julip, og du vil gjerne ha en tesil for å fjærne de minste partikklene fra en drink.\todo{Legg inn bilder}

\subsection{Skjeer}
Skjeer brukes til å røre ting, og måle små mengder med. Ofte er det greit med skjeer med lange skaft, slik at man ikke må lene seg over det man rører. I tilleg kommer enkelte barskjeer vekta, slik at de er enkle å bruke om man skal knuse store isbiter. \todo{bilde}

\subsection{Sitruspresse}
For å få mest mulig saft ut av sitrusfrukter er det enkelt å bruke en sitruspresse. Den gjør det mulig å få mer saft ut av lime og sitron når man trenger saften fra disse fruktene.

\subsection{Muddler}
Muddlere brukes til å mose ting får å frigi smaker fra det man moser. De to beste eksemplene er en klassisk bringebær daiquiri eller en mojito.

\subsection{Glass}
Forskjellige drinker serveres i forskjellige glass. Noen av glassvalgene er for glassets funksjon, mens andre er bare motevalg. Noen av de vanligste glassene man kommer til å møte på er old fashioned glass, highball-glass, og coup, for å nevne noen.

\subsection{Peelere}
Nå beveger vi oss vekk fra hva vi bruker til å lage cocktailer, og over til garnityr (for det meste). Forskjellige peelere brukes til å lage strimler av skallet til forskjellige frukter.

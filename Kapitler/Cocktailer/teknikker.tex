\section{Teknikker}
For å mixe forskjellige drinker er det greit å kunne noen småting. De fleste teknikkene som brukes er enkle, og burde være lette å lære seg. 
\subsection{Shaking}
Kanskje en av de mest gjenkjennbare teknikkene som brukes til cocktail mixing er shaking. Ganske enkelt, ta en cocktailshaker fylt med is og de flytende ingrediensene, sørg for at shakeren er tett, og rist løs. Jeg pleier å riste med shakeren i hodehøyde, men det er mest personlig preferanse. Som oftest shaker man i 8-10 sekunder, men avhengig av hvilken is man bruker, så kan man shake kortere eller lengre.

Shaking brukes som regel for å blande inn sitrus eller annen juice. Det er noen "subtyper" av shaking. Disse brukes for å få litt andre resultater.

\subsubsection{Dryshaking}
Dryshaking gjøres uten is i shakeren. Dette er for å få mer luft mixet inn i det man shaker. Kanskje det beste eksempelet på det er cocktailen Whiskey Sour (oppskrif side \pageref{drink:w_sour}, som har eggehvitte i seg for å lage en annen tekstur på skummet. 

\subsubsection{Whip shake}
Dette er essensielt det samme som en vanlig shake, bare ikke like kraftig. Blir ofte brukt for å raskt kjøle ned noe som skal serveres i knust is.

\subsection{Røring}
Røring brukes for å blande sammen sprit, og samtidig kjøle den ned. Har man god nok is, er det fullt mulig til å få drinken ned til ~-2 grader før servering. Ved å kjenne på temperaturen på utsiden av beholderen, kan man se når man er ferdig med å røre. Når det begynner å dannes kondens på utsiden er man som oftes ferdig.
\subsection{Bygging}
Når man bygger drinker blander man alt i glasset, gjerne uten å røre om. Dette gjøres for det meste for drinker med få ingredienser.

\subsection{Muddling}
Når man muddler moser man ut saft eller oljer fra noe, eller knuser større biter med frukt, sukker etc for å blande det lett sammen.
\subsection{Straining}
Vi siler eller strainer for å fjerne uønskede ting fra cocktailene våres. Dette er som oftest for å ikke ta med den samme isen som er brukt til å shake eller røre, eller for å fjerne biter av frukt eller lignende. 

\subsubsection{Dobble-straining}
Av og til er ikke en julips- eller hawthornestrainer nok til å fjerne alle bitene, så da straines det igjennom en ekstra finmasket sil for å bli kvitt det siste. Dette kan også gjøres for å fjerne små isbiter som har blitt knust vekk fra isen brukt til shakeing.

\subsection{Serveringsmåter}
Det er en rekke forskjellige måter å servere drinker på. I all hovedsak er det kombinasjonen av glass og is som gir utfall her.

\subsubsection{Up}
Uten is i glasset. Noe man kan vær obs på er at drinken gjerne blir varm innen ca 30 min.
\subsubsection{On The Rocks}
Servert med isbiter i glasset.
\subsubsection{Crushed Ice}
Servert med knust is i glasset.
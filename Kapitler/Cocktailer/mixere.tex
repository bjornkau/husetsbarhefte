\section{Mixere}
\subsection{Sirup}
For å legge til noe søtt til cocktailer lages ofte forskjellige siruper. Disse har i all hovedsak rent sukker som hovedsøtstoff, og finnes i mange forskjellige varianter. Eksempler er:
\begin{itemize}
    \item Simple sirup
    \item Honningsirup
    \item Bringebærsirup
    \item Ingefærsirup
    \item Demerarasirup
    \item Invers simple sirup
    \item Grenadin
\end{itemize}

Den viktigste av dem er Simlpesirup, etterfulgt av honning om man lager gamle cocktailer, og forskjellige bærsiruper om man lager nyere cokctailer.

Siruper kom til drinkverden da man gikk fra å ha lunke/varme drinker, til at man lagde kalde drinker med is i. Dette var for å gjøre det enklere å mikse inn sukkeret, som til da hadde vært vanlig sukker. 

\subsection{Oppskrifter}
De følgene oppskriftene er de samme som jeg bruker når jeg mekker cocktailer til meg selv. Noen ting har jeg til gode å finne gode oppskrifter til, så de vil bli lagt til etterhvert.
\subsubsection{Simple sirup}
Bland like store mengder, etter volum, varmt/kokende vann og sukker. Rør ut sukkeret. Må lagres kaldt, da det gir den lengre levetid.
\subsubsection{Honningsirup}
Bland 3 deler honning til 1 del varmt/kokende vann, igjen etter volum. Rør det godt sammen. Må også lagres kaldt.
\subsubsection{Demerarasirup}
Samme som simple, men med demerarasukker istedet.

\subsection{Juice}
Bruken av fruktjuice i drinker er omtrent like gammelt som drinker selv. En Gimlet, som er en av de eldste drinkene, bruker sitronsaft som en stor del av smaksprofilen sin. Forskjellige juicetyper gir forskjellige smaker. De vanligste typene i bruk er sitron og lime, da de er sure, og lett å få tak i. 

Det anbefales på det sterkeste å presse fersk juice der man kan. forskjellen i smak på nypresset juice, kontra juice med konserveringsmiddler er stor. Man kan presse juicen for hånd, eller bruke en juicepresse.

\subsubsection{Sourmix}
Sourmix er en blanding av surt og søtt, man kan bruke ferdig mix som vi gjør på Huset, eller så kan man mikse sitron- og limesaft med simplesirup. Personlig bruker jeg den siste når jeg lager cocktailer selv, da jeg ofte synes det har bedre smak.

\subsection{Mineralvann}
Forskjellige typer mineralvann blir brukt for å tilsette smak, vanne ut, eller tilføre kullsyre til drinker.
\subsubsection{Tonic Water}

\subsubsection{Club Soda}

\subsubsection{Brus}

\subsubsection{Ingefærøl}

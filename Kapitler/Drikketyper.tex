\chapter{Drikketyper}
\section{Definisjoner}
Etter norske lover har vi et lite knippe med definisjoner på hvilke drikketyper vi har. Disse definisjonene er i all hovedsak bestemt i alkoholloven, og gir lite eller ingen info rundt hva det er, eller hva de smaker. De kan brytes ned i styrke på drikken, hvem man har lov til å selge det til, og hvilke skatter og avgifter som er på dem. Drikketypene er delt i fem deler, og de er som følger:
\begin{itemize}
    \item[Alkoholfri]  < 0,7 ABV
    \item[Alkoholsvak] 0,7 - 2,5 ABV 
    \item[Type 1] 2,5 - 4,7 ABV 
    \item[Type 2] 4,7 - 22 ABV
    \item[Type 3] 22 - 60 ABV
\end{itemize}

Det vesentligste å ta med seg fra den lista er at alt som er 22 ABV og lavere er det lov å selge til folk som er over 18 (alkofritt til alle), gitt at de ikke er for beruset. Vi har på vanlige åpningsdager kun bevilgning til å selge opp til Type 2, og en sjelden gang utvidet bevilgning til sterkere salg. \textit{Jeg jobbet på et arrangement som hadde utvidet bevilgning på Huset en kveld. Ila 8 timer med åpning, så solgte vi hele et glass cognac. Mer bortkasta bevilgning har jeg aldri vært borte i - Bjørn.}

Når det gjelder definisjoner som vin, øl, whisk(e)y etc, så blir de diskutert der det passer.